
\begin{center}
\pagebreak
\vspace*{1\baselineskip}
\textbf{\large Conclusion}
\end{center}


This thesis presents a search for hadronic signatures of a strongly-coupled hidden dark sector, accessed via resonant production of a $Z'$ mediator. A model in which the massive $Z'$ mediator decays to two dark quarks which each shower partially back to the visible sector is presented. The resulting shower of dark and Standard Model particles creates a ``semi-visible jet''.\par

The analysis is performed using 139 \fb~of proton-proton collision data at $\sqrt{s}$ = 13 TeV of center-of-mass energy collected by the ATLAS experiment during Run 2 of the LHC. The analysis isolates events with visible and missing energy contributions which are consistent with the semi-visible jet topology. Two machine learning approaches are used to identify events consistent with the hadronic signatures of the hidden dark sector. The first is a supervised permutation invariant classifier approach which models jets according to their associated tracks. The second is a novel anomaly detection approach which couples a permutation invariant event modeling derived from the supervised approach with an unsupervised variational auto-encoder. The anomaly detection approach broadens the sensitivity of the analysis to a wider variety of possible dark jet signatures.\par

The background is estimated via a polynomial fit of the transverse mass spectrum. No significant signal excesses are observed with respect to the expected background processes. Upper limits at the 95\% C.L. on the effective $Z'$ cross section are set for signal models with $m_{Z'}$ ranging from 2000 to 5000 GeV and \rinv~ ranging from 0.2 to 0.8. The anomaly detection approach reports no significant excess, with a maximum observed local significance of 0.877$\sigma$. \par

The results presented here represent exploration of a new phase space in ATLAS Run 2 data. While no evidence of a strongly-coupled hidden dark sector is found, a novel architecture for permutation invariant low-level anomaly detection is presented. As ATLAS continues to collect data in Run 3 and beyond, further explorations of Hidden Valley models which go beyond the scope of this analysis are possible.


%\pagenumbering{gobble}  %remove page number on summary page


