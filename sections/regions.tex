

\section{SVJ Fit and Discovery Analysis Strategies}
\label{sec:strategies}
As was introduced in Chapter~\ref{ch:ml_tools} this analysis is interested in achieving dual goals: to make the best possible measurement of the SVJ signal model generated for this analysis, and to broadly search for any signals consistent with dark QCD behavior and inconsistent with a Standard Model background hypothesis. To this end, two parallel analysis strategies are developed.\par

The SVJ Fit strategy uses the supervised PFN ML score in defining the signal region. Recall, the PFN is trained over simulated MC background and a combination of all MC SVJ signals. This gives this ML tool high sensitivity to the particular nuances of the SVJ shower predicted by the modeled theory. In addition to using the supervised ML tool, the SVJ Fit analysis strategy sets limits on the expected cross section of each signal point in the SVJ signal grid. To achieve this, the shape of the SVJ signals are considered in the final fit, as will be elaborated on Section~\ref{subsec:fit_exclusion}. The combination of the supervised PFN ML score and the signal-shape sensitive fitting strategy allows for the greatest possible sensitivity to the modeled signal process, thus allowing the analysis the best chance at discovery of this model, or enabling the analysis to set the best possible limits on the observed cross section.\par

In contrast, the Discovery analysis strategy attempts to design a more general search, which could be sensitive to SVJs, but also to other possible hidden valley dark QCD models, such as fully dark jets or emerging jets \cite{snowmass}. The Discovery analysis strategy uses the semi-supervised ANTELOPE ML score in defining the signal region. Recall, the ANTELOPE is trained over ATLAS data only, with no explicit knowledge of the SVJ signal behavior. The Discovery fit strategy is also signal model agnostic, by employing a bump hunt \cite{bumphunt} strategy, which searches a smoothly falling template for any bumps inconsistent with a background only hypothesis. Therefore any signal which could present a resonant signature in \mt~could show up as an excess in this strategy. \par

The details of both strategies will be explored in the follow sections which detail the design of the signal regions and fit strategies.
Figure~\ref{fig:fit_strategy} illustrates the difference in the fitting concept.
\begin{figure}[!htbp]
\centering
    \includegraphics[width=0.5\textwidth]{figures/eventsel/fit_strategy}
    \caption{The two fitting strategies. The SVJ Fit (left) illustrates how SVJ signal shapes will be considered in the fit to search for SVJ specific signal shapes, where ``s+b fit'' indicates a fit that considers the shape of the signal. The Discovery Fit (right) illustrates how the data is compared to a background-only hypothesis to search for any kind of \mt~bump, where ``b fit'' indicates a background-only fit with no signal hypothesis.
    \label{fig:fit_strategy}}
\end{figure}

\section{Analysis Regions}
%------------------------------------------------- 
\subsection{Control and Validation Regions}
\label{subec:sel_crvr}

The final background estimation will come from a polynomial fit to the \mt~distribution in the signal region.
The control and validation regions are needed to develop and test this fit in data.
 
To define the CR selection, a variable is needed that isolates background from all signals across the (\rinv, $m_Z$) grid, which is challenging due to the varying nature of the signal models in quantities such as \met~and \pt~, as illustrated in Figure~\ref{fig:presel_vars}. 
The variable \textit{jet width} is chosen, which is the calorimeter measurement of the spread of the clusters which are used to define the jet~\cite{jetwidth}.
The concept is illustrated in Figure~\ref{fig:jet2_calo}.
Jets with only one very energetic cluster have a small width, while jets with many lower energy clusters have a large width.
\begin{figure}[!htbp]
\centering
   \includegraphics[width=0.95\textwidth]{figures/eventsel/jet2_calo}
    \caption{Recall the construction of anti-$k_t$ jets as described in Section~\ref{sec:jet_cluster} and illustrated in Figure~\ref{fig:jet_algorithms}. On the right, we zoom in on two jets, illustrating the narrow cluster pattern in the green jet, and the wide cluster pattern in the yellow jet.
    \label{fig:jet2_calo}}
\end{figure}

Figure~\ref{fig:jet2width} shows jet width specifically for the subleading jet, in data, background MC and signal at preselection.
The leading jet width, which was determined to be less useful for isolating signal from background, is also shown.
The subleading jet is more likely to be aligned with \met, which is why the signal jet width is consistently wider in the subleading jet, but not the leading jet.  %, using v12 of the ntuples.
A selection of width$_{j2} <$ 0.05 is chosen for the CR, with the VR and SR therefore having a selection of width$_{j2}$ $\geq$ 0.05.
 
\begin{figure}[!htbp]
\centering
   %\includegraphics[width=0.4\textwidth]{figures/background/width$_{j2}$_datamc}
   \includegraphics[width=0.98\textwidth]{figures/background/jet2Width}
    \caption{Distributions of the subleading jet width width$_{j2}$ (left) and leading jet width width$_{j1}$ (right) in data, background MC and signals at preselection. All SVJ signals are seen to be wider than the background in width$_{j2}$. The same is not true for width$_{j1}$, where some signals are observed to closely match the background. 
    \label{fig:jet2width}}
\end{figure}

While the CR was used to develop the polynomial strategy, and is the primary region used in many of the fit studies, a validation region is used as an additional check of the estimation strategy in data.
The VR is defined using the region of events with low ML score by either the PFN or ANTELOPE networks.
Here the analysis strategy splits into the two parallel strategies presented in Section~\ref{sec:strategies}: the SVJ fit strategy and the Discovery strategy.
A selection of [PFN score $\leq$ 0.6 \& width$_{j2}$ $\geq$ 0.05] defines the SVJ Fit VR, while [ANTELOPE score $\leq$ 0.7 \& width$_{j2}$ $\geq$ 0.05] defines the discovery VR. 

There are therefore three variables that are crucial to the analysis strategy: width$_{j2}$, ML score, and \mt.
%Figure~\ref{fig:bkg_correlations} shows the correlations of all three variables to one another.
%Any outstanding correations are shown in Figure~\ref{fig:crvrsr_mt} to not sculpt the \mt~distribution and only affect its slope, making these variables trustworthy for extrapolation across background/signal regions and final fitting procedures.
%\begin{figure}[!htbp]
%\centering
 %  \includegraphics[width=0.95\textwidth]{figures/background/bkg_correlations}
 %  \includegraphics[width=0.95\textwidth]{figures/background/bkg_correlations_antelope}
 %   \caption{2D plots revealing correlations between width$_{j2}$ and \mt~(left), width$_{j2}$ and ML score (middle), and \mt~with ML score (right). For the top row, the ML score is the PFN score, and for the bottom three, the ML score is the ANTELOPE score. Minimal correlations are observed and are shown to not sculpt \mt, validating these variables for analysis region construction and statistical treatment.
%    \label{fig:bkg_correlations}}
%\end{figure}
We check the expected shape of \mt~across the CR, VR, and SR using background MC to ensure the shape is smoothly falling across all 3 regions.
Figure~\ref{fig:crvrsr_mt} shows the distribution of \mt~across the CR, VR, and SR, for both the PFN (supervised) and ANTELOPE (semi-supervised) strategies.
No significant bumps or sculpting are observed.
Some slope is observed in the ratio of the CR to the VR/SR shapes; however, the chosen background estimation strategy of polynomial fitting (to be discussed in Section~\ref{sec:background}) is expected to accommodate this slope.
Further, testing the ability of the background polynomial to fit shapes with a variety of slopes increases our confidence in the ability to background polynomial to fit the blinded SR \mt~distribution.%, which could be more problematic for a bump-hunt analysis.
\begin{figure}[!htbp]
\centering
   \includegraphics[width=0.98\textwidth]{figures/eventsel/mT_regions}
    \caption{\mt~in simulation across the CR, VR, and SR for both PFN (left) and ANTELOPE (right) selections. While there is variation in the slope of the distribution, no sculpting of bumps is observed.
    \label{fig:crvrsr_mt}}
\end{figure}

%------------------------------------------------- 
\subsection{Signal Region}
\label{subec:sel_sr}

A selection of PFN score $>$ 0.6 in the SVJ Fit region and ANTELOPE score $>$ 0.7 in the Discovery region is made to provide the primary signal-to-background enrichment, as motivated by Section~\ref{subsec:supervised}.
These values are determined to maximize $s/\sqrt{b}$ in each region.
The additional selection of {width$_{j2}$ $\geq$ 0.05} orthogonalizes the SR to the CR.
Note that the PFN and ANTELOPE regions are not orthogonal; this is because the two analysis flows serve different purposes, their statistical treatments are different, and they will not be combined. 

A summary of the SR, CR, and VR definitions can be seen in Figure~\ref{fig:crvrsr_2d}, along with the relative data statistics in each region.
\begin{figure}[!htbp]
\centering
    \includegraphics[width=0.98\textwidth]{figures/eventsel/crvrsr_2d}
    \caption{Distribution of data events amongst the CR, VR, and SR regions, along with the fractional population of each region. The SVJ Fit region is shown left with the PFN score on the x-axis, and Discovery region is shown right, with the ANTELOPE score on the x-axis.
    \label{fig:crvrsr_2d}}
\end{figure}

A diagram demonstrating the complete analysis flow can be seen in Figure~\ref{fig:analysisflow}.
\begin{figure}[!htbp]
\centering
    \includegraphics[width=1.1\textwidth]{figures/eventsel/analysisflow}
    \caption{Flow of analysis selections and fitting strategy. From preselection, events with Jet2Width < 0.05 are set aside for the CR. Events with Jet2Width $\geq$ 0.05 are split according the ML score. Events with low ML score create the VR, while events with high ML score create the SR. Events with high PFN score are fitted to determine if they are compatible with the SVJ signal shape. Events with high ANTELOPE score are fitted for a background estimation, and a search for any general data bump is performed.
    \label{fig:analysisflow}}
\end{figure}


