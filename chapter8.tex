\chapter{Analysis Strategy}
\label{ch:analysis}

This chapter will present the strategies used to isolate ATLAS data events most consistent with the SVJ model and to estimate the relevant background. The data and MC samples discussed in Chapter~\ref{ch:mc_data} are studied to create the analysis strategy, and the ML scores discussed in Chapter~\ref{ch:ml_tools} are used to isolate the most signal like events. The background is estimated from the \textit{transverse mass} (\mt) spectrum, which captures the main components of the $Z'$ decay. A \textit{preselection} selects events consistent with the SVJ topology based on basic features of the jets and \met. Preselected events are then split into a \textit{control region} (CR), \textit{validation region} (VR), and \textit{signal region} (SR). The CR and VR are used to validate the background estimation procedure. The SR is blinded during the development of the analysis strategy, and only unblinded to make the final measurements presented in Chapter~\ref{ch:results}. The final result is a polynomial fit of the transverse mass spectrum in the SR. The preselection, region definitions, and polynomial fit will be discussed in detail in the following sections.

\section{Transverse Mass}
\label{sec:mt}
The transverse mass \mt~is chosen as the search variable due to the potential for the SVJ signal to create a resonant shape around the mass of the $Z'$. \mt~is the total transverse mass of the two leading\footnote{Recall the definition of the leading jets from Chapter~\ref{ch:mc_data}, meaning the two jets with highest \pt~ in the event.} jets and the \met, expressed in Equation~\ref{eq:mt} as:

\begin{equation}
m_T^2 = [E_{T,jj} + \met~]^2 - [\vec{p}_{T,jj} + \vec{p}_T^{\text{miss}}]^2
\label{eq:mt}
\end{equation}

where $E_{T,jj}$ is the transverse energy of the dijet system created by the two leading jets. We take $E_{T,jj} = m_{jj}^2 + |\vec{p}_{T,jj}|^2$, where $m_{jj}^2$ is the invariant mass of the two leading jets, and $\vec{p}_{T,jj}$ is the vector sum of the \pt~of the two leading jets. \mt~is selected as the search variable in place of simpler invariant mass $m_{jj}$ because substantial energy from the Z' decay is captured in the \met~. \par

Figure~\ref{fig:mt_mass} illustrates the resonance in \mt~of the SVJ signals. The smoothly falling background is shown in comparison to the resonant shape of the signals, which form a peak just below the associated $Z'$ mass. The lower \rinv~signals form a more distinctive resonance in \mt, while the high \rinv~signals have a much wider \mt~shape.
\begin{figure}[!htbp]
\centering
    \includegraphics[width=0.98\textwidth]{figures/ch8/mt_mass_norm}
    \includegraphics[width=0.98\textwidth]{figures/ch8/mt_mass_unnorm}
    \caption{The resonant shape of the SVJ signals (color) in \mt, in contrast to the smoothly falling \mt~background (grey). The top row illustrates unit normal shapes, so that the shape of the signals is more easily seen. The bottom row illustrates the signal and background scaled to their expected yield at preselection, illustrating the relative expected statistics. The \rinv~= 0.8 signals (right) boast a wider shape, making them more difficult to detect, while the \rinv~= 0.2 signals (left) produce a more narrow resonance in \mt~. The signal models are identified in the legend as ``$m_{Z'}$, \rinv''. 
    \label{fig:mt_mass}}
\end{figure}

%------------------------------------------------- 
\section{Preselection}
\label{sec:eventsel}

The preselection isolates the phase space of events that most closely match the SVJ signal topology. Each cut was determined to reduce the background and enhance signal sensitivity.
The list of preselection cuts and the motivation behind each cut are as follows. 
Here ``jets" refer to anti-$k_t$ R=0.4 jets, as discussed in Chapter~\ref{ch:part_reco}.
Table~\ref{fig:presel_cutflow} shows the impact of these cuts in sequence for data and signal.

\begin{itemize}
\item At least 2 jets; in order to reconstruct the resonance mass
\item Leading jet ($j_1$) \pt $>$ 450 GeV; to ensure the trigger is fully efficient
\item Subleading jet ($j_2$) \pt $>$ 150 GeV; to mitigate the presence of non-collision background (Appendix~\ref{app:ncb})
\item $|\eta_{j1,j2}|$ $<$ 2.1; to ensure jets are fully within the tracker
\item $\Delta$Y$<$ 2.8 (difference in rapidity between $j_1$ and $j_2$); to ensure central production associated with the hard scatter  
\item \met $>$ 200 GeV; to restrict the phase space to events with possible dark particles 
\item \mt $>$ 1.2 TeV, to ensure a smoothly falling \mt~distribution for fitting (Section~\ref{sec:background})
\item At least 3 tracks for each of the two leading jets $j_1$ and $j_2$; to have adequate tracking information for the ML tools
\item $\Delta\Phi$($j_1$,$j_2$) $>$ 0.8; to mitigate the presence of non-collision background (Appendix~\ref{app:ncb}).
\end{itemize}

\begin{table}[!htbp]
\centering
   \includegraphics[width=0.95\textwidth]{figures/eventsel/preselection/presel_cutflow}
    \caption{Preselection cuts for data (left) and signal (right).
    \label{fig:presel_cutflow}}
\end{table}

With the exception of the cuts necessary to reduce the non-collision background, all cuts were verified to enhance signal sensitivity by improving $s/\sqrt{b}$, a standard estimate of discovery sensitivity, where $s$ is the number of signal events and $b$ is the number of background events. The cuts on $\Delta$Y and \met~were optimized to enhance $s/\sqrt{b}$, and the other cuts were informed by the physics motivations provided above. \par

Vetos are applied to reject any events where an error for a subdetector is flagged. 
To reject non-collision backgrounds (NCB), such as calorimeter noise, beam halo interactions, or cosmic rays, the standard ATLAS event cleaning procedure is applied.
As this analysis is very dependent on \met~associated to jets, the \textsc{Tight}~\cite{tight_loose} event cleaning working point is applied. 
Tight cleaning requires jets to pass a stricter set of quality requirements compared to the \textsc{Loose}~\cite{tight_loose} cleaning option.
Due to the alignment between jets and \met~ for SVJ events, it was found that two additional cuts (indicated above) are needed to remove NCB.
The process for selecting these cuts is presented in Appendix~\ref{app:ncb}. 

The leading and subleading jets in each event are considered the dark quark candidates from the $Z' \rightarrow q_D\bar{q_D}$ decay.  
They are therefore the two jets of greatest interest in the event, and used in the computation of key analysis variables.
This choice was determined through studies of the dark quark trajectory in simulation which determined that the leading and subleading jets are most often aligned with the dark quarks, and therefore most likely to capture the dark quark hadronization.
This study can be found in Appendix~\ref{app:truthstudies}.

Figure~\ref{fig:presel_vars} and Figure~\ref{fig:presel_vars2} show the distribution of signals, data and background MC in several key analysis variables after preselection is applied.
The variables illustrated are:
\begin{itemize}
\item Transverse mass \mt: key analysis variable which reconstructs the $Z'$ mass, as discussed in Section~\ref{sec:mt}.
\item Leading jet \pt: the trigger variable, and a component of \mt. 
\item Subleading jet \pt: dark quark candidate and component of \mt.
\item Missing transverse energy \met~(or MET): component of \mt, and an indication of the presence of dark hadrons. All signals are observed to have more \met~than the background.
\item $\Delta\phi$(j1, j2): difference in trajectory of the two leading jets, measured in the $\phi$ plane (recall the ATLAS detector geometry of Figure~\ref{fig:ATLASgeometry}). Orientation of the jets is of importance to the ML model as discussed in Section~\ref{sec:input_model}. 
\item $\Delta$Y(j1, j2): difference in trajectory of the two leading jets, measured in the Y plane (recall Figure~\ref{fig:ATLASgeometry} and the definition of rapidity Equation~\ref{eq:rapidity}). The signals are seen to have lower $\Delta$Y(j1, j2) than the background.
\item $\Delta\phi$(j1, \met): the angular separation between the leading jet and the \met. The leading jet is predominantly back-to-back with the \met.
\item $\Delta\phi$(j2, \met): the angular separation between the subleading jet and the \met. The subleading jet is predominantly aligned with the \met, which is a unique feature of this analysis as jets that are closely aligned with \met~are often removed from other ATLAS analyses.
\end{itemize}

\begin{figure}[!htbp]
\centering
    \includegraphics[width=0.95\textwidth]{figures/eventsel/preselection/presel1}
    \includegraphics[width=0.95\textwidth]{figures/eventsel/preselection/presel2}
    \caption{Energy and momentum analysis variables at preselection, for data (black), background MC (grey), and representative signal models (color). The signal models are identified in the legend as ``$m_{Z'}$, \rinv''. 
    \label{fig:presel_vars}}
\end{figure}

\begin{figure}[!htbp]
\centering
    \includegraphics[width=0.95\textwidth]{figures/eventsel/preselection/presel3}
    \includegraphics[width=0.95\textwidth]{figures/eventsel/preselection/presel4}
     \caption{Orientation analysis variables at preselection, for data (black), background MC (grey), and representative signal models (color). The signal models are identified in the legend as ``$m_{Z'}$, \rinv''. 
      \label{fig:presel_vars2}}
\end{figure}

The data and background MC are both illustrated in Figure~\ref{fig:presel_vars} and Figure~\ref{fig:presel_vars2}. The agreement between them is generally observed to be good, particularly in the key analysis variable \mt. The agreement is not required to be perfect as the background MC is not used for the background estimation. The primary motivation for studying the background MC is to uncover and remove issues unique to data such as the non-collision background.

 \clearpage
%\input{sections/object}


\section{SVJ Fit and Discovery Analysis Strategies}
\label{sec:strategies}
As was introduced in Chapter~\ref{ch:ml_tools} this analysis is interested in achieving dual goals: to make the best possible measurement of the SVJ signal model generated for this analysis, and to broadly search for any signals consistent with dark QCD behavior and inconsistent with a Standard Model background hypothesis. To this end, two parallel analysis strategies are developed.\par

The SVJ Fit strategy uses the supervised PFN ML score in defining the signal region. Recall, the PFN is trained over simulated MC background and a combination of all MC SVJ signals. This gives this ML tool high sensitivity to the particular nuances of the SVJ shower predicted by the modeled theory. In addition to using the supervised ML tool, the SVJ Fit analysis strategy sets limits on the expected cross section of each signal point in the SVJ signal grid. To achieve this, the shape of the SVJ signals are considered in the final fit, as will be elaborated on Section~\ref{subsec:fit_exclusion}. The combination of the supervised PFN ML score and the signal-shape sensitive fitting strategy allows for the greatest possible sensitivity to the modeled signal process, thus allowing the analysis the best chance at discovery of this model, or enabling the analysis to set the best possible limits on the observed cross section.\par

In contrast, the Discovery analysis strategy attempts to design a more general search, which could be sensitive to SVJs, but also to other possible hidden valley dark QCD models, such as fully dark jets or emerging jets \cite{snowmass}. The Discovery analysis strategy uses the semi-supervised ANTELOPE ML score in defining the signal region. Recall, the ANTELOPE is trained over ATLAS data only, with no explicit knowledge of the SVJ signal behavior. The Discovery fit strategy is also signal model agnostic, by employing a bump hunt \cite{bumphunt} strategy, which searches a smoothly falling template for any bumps inconsistent with a background only hypothesis. Therefore any signal which could present a resonant signature in \mt~could show up as an excess in this strategy. \par

The details of both strategies will be explored in the follow sections which detail the design of the signal regions and fit strategies.
Figure~\ref{fig:fit_strategy} illustrates the difference in the fitting concept.
\begin{figure}[!htbp]
\centering
    \includegraphics[width=0.5\textwidth]{figures/eventsel/fit_strategy}
    \caption{The two fitting strategies. The SVJ Fit (left) illustrates how SVJ signal shapes will be considered in the fit to search for SVJ specific signal shapes, where ``s+b fit'' indicates a fit that considers the shape of the signal. The Discovery Fit (right) illustrates how the data is compared to a background-only hypothesis to search for any kind of \mt~bump, where ``b fit'' indicates a background-only fit with no signal hypothesis.
    \label{fig:fit_strategy}}
\end{figure}

\section{Analysis Regions}
%------------------------------------------------- 
\subsection{Control and Validation Regions}
\label{subec:sel_crvr}

The final background estimation will come from a polynomial fit to the \mt~distribution in the signal region.
The control and validation regions are needed to develop and test this fit in data.
 
To define the CR selection, a variable is needed that isolates background from all signals across the (\rinv, $m_Z$) grid, which is challenging due to the varying nature of the signal models in quantities such as \met~and \pt~, as illustrated in Figure~\ref{fig:presel_vars}. 
The variable \textit{jet width} is chosen, which is the calorimeter measurement of the spread of the clusters which are used to define the jet~\cite{jetwidth}.
The concept is illustrated in Figure~\ref{fig:jet2_calo}.
Jets with only one very energetic cluster have a small width, while jets with many lower energy clusters have a large width.
\begin{figure}[!htbp]
\centering
   \includegraphics[width=0.95\textwidth]{figures/eventsel/jet2_calo}
    \caption{Recall the construction of anti-$k_t$ jets as described in Section~\ref{sec:jet_cluster} and illustrated in Figure~\ref{fig:jet_algorithms}. On the right, we zoom in on two jets, illustrating the narrow cluster pattern in the green jet, and the wide cluster pattern in the yellow jet.
    \label{fig:jet2_calo}}
\end{figure}

Figure~\ref{fig:jet2width} shows jet width specifically for the subleading jet, in data, background MC and signal at preselection.
The leading jet width, which was determined to be less useful for isolating signal from background, is also shown.
The subleading jet is more likely to be aligned with \met, which is why the signal jet width is consistently wider in the subleading jet, but not the leading jet.  %, using v12 of the ntuples.
A selection of width$_{j2} <$ 0.05 is chosen for the CR, with the VR and SR therefore having a selection of width$_{j2}$ $\geq$ 0.05.
 
\begin{figure}[!htbp]
\centering
   %\includegraphics[width=0.4\textwidth]{figures/background/width$_{j2}$_datamc}
   \includegraphics[width=0.98\textwidth]{figures/background/jet2Width}
    \caption{Distributions of the subleading jet width width$_{j2}$ (left) and leading jet width width$_{j1}$ (right) in data, background MC and signals at preselection. All SVJ signals are seen to be wider than the background in width$_{j2}$. The same is not true for width$_{j1}$, where some signals are observed to closely match the background. 
    \label{fig:jet2width}}
\end{figure}

While the CR was used to develop the polynomial strategy, and is the primary region used in many of the fit studies, a validation region is used as an additional check of the estimation strategy in data.
The VR is defined using the region of events with low ML score by either the PFN or ANTELOPE networks.
Here the analysis strategy splits into the two parallel strategies presented in Section~\ref{sec:strategies}: the SVJ fit strategy and the Discovery strategy.
A selection of [PFN score $\leq$ 0.6 \& width$_{j2}$ $\geq$ 0.05] defines the SVJ Fit VR, while [ANTELOPE score $\leq$ 0.7 \& width$_{j2}$ $\geq$ 0.05] defines the discovery VR. 

There are therefore three variables that are crucial to the analysis strategy: width$_{j2}$, ML score, and \mt.
%Figure~\ref{fig:bkg_correlations} shows the correlations of all three variables to one another.
%Any outstanding correations are shown in Figure~\ref{fig:crvrsr_mt} to not sculpt the \mt~distribution and only affect its slope, making these variables trustworthy for extrapolation across background/signal regions and final fitting procedures.
%\begin{figure}[!htbp]
%\centering
 %  \includegraphics[width=0.95\textwidth]{figures/background/bkg_correlations}
 %  \includegraphics[width=0.95\textwidth]{figures/background/bkg_correlations_antelope}
 %   \caption{2D plots revealing correlations between width$_{j2}$ and \mt~(left), width$_{j2}$ and ML score (middle), and \mt~with ML score (right). For the top row, the ML score is the PFN score, and for the bottom three, the ML score is the ANTELOPE score. Minimal correlations are observed and are shown to not sculpt \mt, validating these variables for analysis region construction and statistical treatment.
%    \label{fig:bkg_correlations}}
%\end{figure}
We check the expected shape of \mt~across the CR, VR, and SR using background MC to ensure the shape is smoothly falling across all 3 regions.
Figure~\ref{fig:crvrsr_mt} shows the distribution of \mt~across the CR, VR, and SR, for both the PFN (supervised) and ANTELOPE (semi-supervised) strategies.
No significant bumps or sculpting are observed.
Some slope is observed in the ratio of the CR to the VR/SR shapes; however, the chosen background estimation strategy of polynomial fitting (to be discussed in Section~\ref{sec:background}) is expected to accommodate this slope.
Further, testing the ability of the background polynomial to fit shapes with a variety of slopes increases our confidence in the ability to background polynomial to fit the blinded SR \mt~distribution.%, which could be more problematic for a bump-hunt analysis.
\begin{figure}[!htbp]
\centering
   \includegraphics[width=0.98\textwidth]{figures/eventsel/mT_regions}
    \caption{\mt~in simulation across the CR, VR, and SR for both PFN (left) and ANTELOPE (right) selections. While there is variation in the slope of the distribution, no sculpting of bumps is observed.
    \label{fig:crvrsr_mt}}
\end{figure}

%------------------------------------------------- 
\subsection{Signal Region}
\label{subec:sel_sr}

A selection of PFN score $>$ 0.6 in the SVJ Fit region and ANTELOPE score $>$ 0.7 in the Discovery region is made to provide the primary signal-to-background enrichment, as motivated by Section~\ref{subsec:supervised}.
These values are determined to maximize $s/\sqrt{b}$ in each region.
The additional selection of {width$_{j2}$ $\geq$ 0.05} orthogonalizes the SR to the CR.
Note that the PFN and ANTELOPE regions are not orthogonal; this is because the two analysis flows serve different purposes, their statistical treatments are different, and they will not be combined. 

A summary of the SR, CR, and VR definitions can be seen in Figure~\ref{fig:crvrsr_2d}, along with the relative data statistics in each region.
\begin{figure}[!htbp]
\centering
    \includegraphics[width=0.98\textwidth]{figures/eventsel/crvrsr_2d}
    \caption{Distribution of data events amongst the CR, VR, and SR regions, along with the fractional population of each region. The SVJ Fit region is shown left with the PFN score on the x-axis, and Discovery region is shown right, with the ANTELOPE score on the x-axis.
    \label{fig:crvrsr_2d}}
\end{figure}

A diagram demonstrating the complete analysis flow can be seen in Figure~\ref{fig:analysisflow}.
\begin{figure}[!htbp]
\centering
    \includegraphics[width=1.1\textwidth]{figures/eventsel/analysisflow}
    \caption{Flow of analysis selections and fitting strategy. From preselection, events with Jet2Width < 0.05 are set aside for the CR. Events with Jet2Width $\geq$ 0.05 are split according the ML score. Events with low ML score create the VR, while events with high ML score create the SR. Events with high PFN score are fitted to determine if they are compatible with the SVJ signal shape. Events with high ANTELOPE score are fitted for a background estimation, and a search for any general data bump is performed.
    \label{fig:analysisflow}}
\end{figure}



\section{Background Estimation}
\label{sec:background}

%Backgrounds: The backgrounds should be evaluated and this should include CR/VR plots with the full data (full run-2 analyses) or at least a representative majority of the data (analyses during data-taking); 
%exceptionally a minor background could be still under finalization, but in this case a short timescale should be envisaged for its completion, or it should be a background that does not affect the accuracy of the result. ( a 10% background on a 10% accuracy measurement is not a minor background)

%This is done via a data template for the shape of \mt~taken from a CR that is orthogonal to the SR, but still close in SM process contribution and kinematic phase space. 
%A polynomial fit is then performed to describe the shape of \mt~in the SR.
%The polynomial is constructed using the CR data template, and validated to data in a VR that is similarly orthogonal to both the CR and SR.
The SM background in the SR is predominantly composed of QCD events, and due to the poor modeling of QCD at high energies by MC, it is estimated in a fully data-driven way. 
An empirical functional form is used for the background shape of \mt.
The ability of this function to model the background behavior is tested both the CR and the VR for each analysis strategy. The shape parameters are left free in all the fits.

The fits are performed for 1500 GeV $<$ \mt~ $<$ 6000 GeV.
The polynomial chosen is a standard 5-parameter function used in several similar dijet search analyses such as \cite{darkjets} \cite{smooth_bkg} \cite{cms_svj} and shown in Equation~\ref{eq:bkgpoly}:
\begin{equation}
f(x) = p_1(1-x)^{p_2}x^{p_3+p_4 lnx+p_5ln^2x}
\label{eq:bkgpoly}
\end{equation}
Here x = m$_{T}$/$\sqrt{s}$ (transverse mass scaled to the $pp$ collision center of mass energy) and $p_i$ are free parameters.
The fit function is required to be fully positive, and the \mt~distribution is fit to 90 even-width bins.
The resulting fit shape is used as the background estimation for both the SVJ Fit strategy and the Discovery strategy. 
Validation of the fit and its ability to both model the background and detect signal are shown in Section~\ref{sec:fit_strategy}.

Higher order polynomials were also considered, but an F-test was performed according to the recommendations in Ref.~\cite{smooth_bkg}, and the five parameter function was determined to be adequate and optimal for capturing the shape of the background.
The F-test uses the test statistic 
\begin{equation}
F = \frac{\frac{\chi_{\text{nom}}^2 - \chi_{\text{alt}}^2}{n_\text{alt} - n_\text{nom}}}{\frac{ \chi_{\text{alt}}^2}{n-n_\text{alt} }}
\end{equation}
where $\chi_{\text{nom}}^2$ and $\chi_{\text{alt}}^2$ are the $\chi^2$ values for the polynomial data fits using the nominal (in this case, 5 parameter) model and the alternate model (in this case, 6 or 7 parameter model).
$n_\text{nom}$ and $n_\text{alt}$ are the number of free parameters in the nominal or alternate model, and $n$ is the number of bins used for the fit.
In the asymptotic limit $F$ follows the Fisher-Snedecor distribution, so that $F$ can be converted to a $p$-value $p(F)$ to determine if the alternate model provides a better fit to the data, indicated by $p(F) < 0.05$.
We find $p(F) >> 0.05$ for a 6 and or 7 parameter polynomial function, indicating there is no improvement from increasing to 6 or 7 parameters.
Comparing a nominal model with 4 parameters to the 5 parameter model, we find $p(F) = 0.013$, indicating that the 5 parameter model is preferable to the 4 parameter model.







\section{Fit Strategy and Validation}
\label{sec:fit_strategy}

The steps taken to validate the fitting approach for both the SVJ Fit strategy and the Discovery strategy will be outlined in the following sections. The signal region fits which comprise the final result will be presented in Chapter~\ref{ch:results}.

\subsection{SVJ Fit Strategy}
\label{subsec:fit_exclusion}

The ability of the five parameter fit function to capture the shape of the background is studied extensively, using data from the CR and VR. Signal injection tests are performed to determine the ability of the fit to recover and quantify any SVJ signal excess. Estimates of the expected sensitivity and the ability to set upper limits on the cross-section of the signal process are also verified.\par
%Results: An overview of the final fit setup including the final discriminating variables(s), the (SR/CR) regions to be included in the fit and the floating normalization parameters. 
%Some rough first expected limits/discovery sensitivity plots are useful if you have them but not necessary. In this case the binning of the final variable(s) and the systematics smoothing/pruning should be indicated.

The fit results are primarily evaluated by their $p$-value, which dictates the probability of observing the given data spectrum given the fit hypothesis.
A higher $p$-value is an indication of better agreement between the data and the fitted shape. 
The $\chi^2$/d.o.f. (or chi-sqaure per degrees of freedom, shown as just ``$\chi^2$'' or ``x'') is also presented.
The $\chi^2$ is checked to make sure it is not substantially larger or smaller than 1.0. 
$\chi^2$ values close to 1.0 indicate that the fit is able to capture the data without overfitting.
For each fit, the pattern of the residuals is also shown.
The residual is calculated as the difference between the observed data in a bin and the fit estimation for the bin, divided by the statistical uncertainty, indicating the significance of the deviation from the fit estimation.

%------------------------------------------------- 
\subsubsection{Background Only Fits}
\label{subsec:fit_bkgonly}

The background fit polynomial is validated using the original data from the CR and VR, and pseudo-data generated from the CR.

The nature of the functional fitting method allows the fit to easily adapt to changes in slope of a smoothly falling distribution.
Thus validation of the fit can be performed in data using the CR and the VR distributions to model the expected behavior in the SR. 
Additionally, if we recall Figure~\ref{fig:crvrsr_mt} which illustrated the difference in the slope of \mt~between the analysis regions, we recall that the greatest difference in slope is expected between the CR and VR.
The expected shape of the SR lies between the CR and the VR, so fitting both of these regions gives us confidence in our ability to fit the SR when it is unblinded.
Figure~\ref{fig:bkgfit_data_fullstats} shows the a successful fit performed on the full statistics CR and VR regions.
The \mt~spectrum is fit in 90 bins of width 50 GeV. 
For the purposes of this analysis, any fit with a $p$-value > 0.05 is considered successful. 

\begin{figure}[!htbp]
\centering
   \includegraphics[width=0.8\textwidth]{figures/stats/bkgfit_data_fullstats}
    \caption{Background-only \mt~fits using data in the full statistics CR and VR regions. The fit is observed to converge with $p$-value > 0.05. The distribution of residuals is reasonably flat. The number of events in the data histogram, $p$-value and $\chi^2$ value (x) are reported in the legend. 
    \label{fig:bkgfit_data_fullstats}}
\end{figure}

Table~\ref{fig:postfit_param_pfn} shows the post-fit values of the fit parameters and their uncertainties for each fit. Fits of the MC background in the CR, VR, and SR were also performed and observed to be successful. These fit are available in Appendix~ref{app:mcfit}. 
\begin{table}[!htbp]
\centering
   \includegraphics[width=0.75\textwidth]{figures/stats/postfit_param_pfn}
    \caption{Post-fit parameters for the PFN CR and VR. $p1$ can also be considered $N_{bkg}$ or the normalization factor.
    \label{fig:postfit_param_pfn}}
\end{table}

To further validate the fit stability of the fit against potential statistical fluctuations, \textit{pseudo-data} (also known as \textit{toy datasets}) are created from the CR data distribution. 
The pseudo-data is created following an \textit{Asimov} prescription \cite{asimov}, using a template to generate a set of toys representing different possible statistical fluctuations.
When studied as a group, the performance of the pseudo-data collection represents the range of possible behavior for an unknown distribution (the SR data in this case), given its statistical uncertainties.

The template used to generate the pseudo-data is a \textit{smoothed} and \textit{scaled} version of the CR. 
The smoothing applied follows the procedure for functional decomposition described in Ref.~\cite{edgar2018functional}.
Figure~\ref{fig:smoothing} shows the impact of smoothing on the source data distribution in the CR.
\begin{figure}[!htbp]
\centering
   \includegraphics[width=0.95\textwidth]{figures/stats/smoothing}
    \caption{\mt~distribution in the data CR, before (left) and after (right) smoothing.
    \label{fig:smoothing}}
\end{figure}

The scaling adjusts the statistics of the smoothed template to the expected statistics of the SR.
Recall Figure~\ref{fig:crvrsr_2d}, which illustrates that the statistics (or number of events) in the CR and the VR are almost 3x the expected statistics of the SR.
The polynomial fitting strategy is sensitive to the statistics of the fitted template, so its performance can vary substantially depending on the statistical power of the fitted distribution.
To simulate this, the smoothed template is scaled to the expected statistics of the SR.
Toys are then generated from the smoothed distribution, by varying each bin within its statistical uncertainty according to a Poisson distribution. 
Each toy has the same statistical power as the SR, within statistical uncertainly.

Figure~\ref{fig:bkgfit_data} shows example fits to three such toy datasets.
Figure~\ref{fig:asimov_hist} shows the resulting p-values after an ensemble of 100 Asimov pseudo-datasets are each individually fit. 
This test determines the likelihood of exceptionally good (high $p$-value) or poor (low $p$-value) fits due to randoms statistical fluctuations in the data. 
A flat distribution is observed, indicating good statistical behavior. 

\begin{figure}[!htbp]
\centering
   %\includegraphics[width=0.32\textwidth]{figures/stats/dataDSfiveParFitChi2_CR0.png}
   %\includegraphics[width=0.32\textwidth]{figures/stats/dataDSfiveParFitChi2_CR1.png}
   %\includegraphics[width=0.32\textwidth]{figures/stats/dataDSfiveParFitChi2_CR2.png}
   %\includegraphics[width=0.32\textwidth]{figures/stats/dataDSfiveParFitChi2_VR0.png}
   %\includegraphics[width=0.32\textwidth]{figures/stats/dataDSfiveParFitChi2_VR1.png}
   %\includegraphics[width=0.32\textwidth]{figures/stats/dataDSfiveParFitChi2_VR2.png}
   \includegraphics[width=0.97\textwidth]{figures/stats/bkgfit_data_cr}
    \caption{Background-only \mt~fits using pseudo-data from the CR template. All three fits are seen to successful converge, with varying $p$-values. The distribution of residuals is reasonably flat for all three fits.
     \label{fig:bkgfit_data}}
\end{figure}

\begin{figure}[!htbp]
\centering
   \includegraphics[width=0.6\textwidth]{figures/stats/asimov_cr_hist}
    \caption{$p$-value histograms from 100 fits to Asimov data in the CR. 98 $p$-values are shown, as two are excluded due to fits that did not converge on the first try. These fits later converged after the initial parameters were adjusted. %, demonstrating a flat shape across p-value.
    \label{fig:asimov_hist}}
\end{figure}

\clearpage
%------------------------------------------------- 
\subsubsection{Signal + Background Fits}
\label{subsec:fit_splusb}

Figure~\ref{fig:splusb_sigInj} shows an example of an injected signal into the exclusion region \mt~spectrum, and the ability of the fit framework to accurately fit the number of signal events.
\begin{figure}[!htbp]
\centering
   \includegraphics[width=0.6\textwidth]{figures/stats/splusb_sigInj}
    \caption{Example S+B fit on a background \mt~spectrum with injected signal from the point (4000 GeV, \rinv=0.2). The shape of the injected signal can be seen in Figure~\ref{fig:mt_mass}. The ability of the s+b fit to capture the shape of the signal and accurately measure the amount of injected signal is observed.
    \label{fig:splusb_sigInj}}
\end{figure}

Signal injection tests demonstrate the a linear relationship between the amount of signal injected and the amount of signal measured by the fit.
The signal injection tests are performed in Asimov datasets to counter the impact of statistical fluctuations in any given template.
50 Asimov trials are run for all signal points across Z' mass and \rinv.

%Bias within the Asimov signal injection test could indicate a lack of sufficient normalization uncertainties on the signal model. 
%In the event of systematic mismeasurement in signal injection, the spurious signal uncertainty as derived from Loose-not-tight or additional uncertainties will be considered.

Figure~\ref{fig:siginj_asimov} provides the results of these tests. 
The uncertainty of the measurement varies according to the Z' mass, due to the larger relative background for lower mass points (larger error-bars for lower Z' masses). 
However, a strong linear relationship between amount of signal injected and amount of signal measured is observed for all signal points, which is the key feature.
The dashed lines illustrate the linear relationship, showing that for all points, the amount of signal measured increases as more signal is injected.
The variation in the y-intercept of the fitted dashed lines and the exact number of signal events measured for each injection level is not as significant as the overall linear behavior which is exhibited.
\begin{figure}[!htbp]
\centering
   \includegraphics[width=0.45\textwidth]{figures/stats/siginj_asimov_02}
   \includegraphics[width=0.45\textwidth]{figures/stats/siginj_asimov_04}
   \includegraphics[width=0.45\textwidth]{figures/stats/siginj_asimov_06}
   \includegraphics[width=0.45\textwidth]{figures/stats/siginj_asimov_08}
   \caption{Measured signal at a variety of injected values (1x, 2x, and 5x $\sqrt{b}$), for all signal points in the grid, \rinv=0.2 (top left), 0.4 (top right), 0.6 (bottom left), and 0.8 (bottom right). The x-axis values are slightly shifted from their true value so that all points can be viewed simultaneously. The error bars indicate the standard deviation of the number of fitted events across the 50 Asimov experiments. While the errors are large for some points, the strong linear relationship of the means, illustrated by the dashed lines, is the key feature.
%he requirement relative to $\sigma_{\text{fit}}$ is met for every signal point and injected signal size, thus satisfying the OR criteria from Equation~\ref{eq:spursig}.
    \label{fig:siginj_asimov}}
\end{figure}

\clearpage
%------------------------------------------------- 
\subsubsection{Expected Sensitivity}
\label{subsec:fit_expsens}

Limits on the signal process are obtained by determining the cross section of the signal that can be excluded at the 95\% Confidence Level (CL). 
\textit{Limits} refer to determining the maximum (or \textit{limiting}) signal cross-section compatible with the observed data spectrum, such that any theory resulting in a signal cross-section above the limit is excluded with 95\% confidence. 
Figure~\ref{fig:limits_exp_1D} shows the expected limits obtained from an average of 50 Asimov data fits. 
The fits are signal + background fits performed on a background-only spectrum, which allows the fit to determine the level of signal compatible with the background-only hypothesis.
The limits shown include a systematic uncertainty on the yield of the signal, arising from the \textit{spurious signal}\footnote{Spurious signal is the amount of signal measured by the fit in the absence of injected signal.} which will be discussed in Section~\ref{sec:syst}.
%Figure~\ref{fig:limits_exp_1D_asimov} shows the expected limits obtained from an average of 50 Asimov toys thrown from the CR.
%The limits are very stable across individual data fits and Asimov, as well as across the CR and VR.
%The alignment of fluctuations between the single CR fit and Asimov toys indicates that the particular shape of data in the CR influences the shape of the limits.
%The limits shown come from an average of 10 Asimov pseudodata fits of the CR. Figure~\ref{fig:perc_success_limit} shows the percentage of Asimov limit tests that result in a successful fit. 

Considerable exclusion power is predicted for low \rinv~signal points and lower mass points.
Higher \rinv~points present more difficulty due to the very broad signal bump.
Higher Z' mass points are more difficult to exclude due to the low theory cross-sections.
\begin{figure}[!htbp]
\centering
   \includegraphics[width=0.95\textwidth]{figures/stats/limits_exp_1D}
    \caption{95\% C.L. upper limits on the $Z'$ production cross-section. All signal models across Z' mass and four different \rinv~fractions are shown. Limits are derived from the shape of \mt~ in the CR region. 
    \label{fig:limits_exp_1D}}
\end{figure}
%\begin{figure}[!htbp]
%\centering
%   \includegraphics[width=0.9\textwidth]{figures/stats/limits_exp_1D_asimov_1}
%   \includegraphics[width=0.9\textwidth]{figures/stats/limits_exp_1D_asimov_2}
%    \caption{95\% C.L. upper limits for signal models across Z' mass, for four different \rinv~fractions, using an average of 50 Asimov pseudo-data tests from the CR (left) and VR (right) (without systematics).
%    \label{fig:limits_exp_1D_asimov}}
%\end{figure}

%\begin{figure}[!htbp]ß
%\centering
%   \includegraphics[width=0.6\textwidth]{figures/stats/perc_success_limit}
%    \caption{Percent of Asimov pseudodata S+B fits with successful fit and successful limit convergence.
%    \label{fig:perc_success_limit}}
%\end{figure}

The ability of the fit to identify a significant excess is tested by calculating the limits on signal injected toys. $2\sigma$ and $5\sigma$ of signal is injected for each signal point into 50 Asimov data toys.
%The number of signal events necessary for a $2\sigma$/ $5\sigma$ excess is calculated for each signal point from the expected limits in the background-only case shown in Figure~\ref{fig:limits_exp_1D}.
%The expected limit represents the limit on a $2\sigma$ excess, so a $5\sigma$ excess requires 2.5x as much signal.
Figure~\ref{fig:lim_sig_inj} demonstrates the impact of this signal injection on the limit for \rinv~= 0.2.
The observed limit rises as more signal is injected, indicating the ability of the fit to identify a significant signal excess. 

\begin{figure}[!htbp]
\centering
   \includegraphics[width=0.98\textwidth]{figures/stats/lim_sig_inj}
    \caption{95\% C.L. upper limits and observed limit for signal models across Z' mass, with varying amounts of signal injected. The increasing observed limit indicates the desired behavior.
    \label{fig:lim_sig_inj}}
\end{figure}

\clearpage

%------------------------------------------------- 
\subsection{Discovery Strategy)}
\label{subsec:fit_bh}

Model-independent fits for the discovery region will be performed using \href{https://github.com/scikit-hep/pyBumpHunter}{pyBumpHunter}.
The strategy will consist of comparing the data in a given \mt~spectrum of interest to a background estimation derived by performing the polynomial fit and sampling from the post-fit function into a histogram.
The polynomial fit is done to an \mt~distribution with 180 bins (25 GeV wide). %90 bins, as with the PFN \mt~spectrum.
To keep the trials factor moderate, a rebinning will be performed based on the signal mass resolution in \mt (Section~\ref{subsec:binning}) to best assess the significance of BumpHunter results.
This is under development with preliminary studies shown in Appendix~\ref{app:bumphunter}.

Figure~\ref{fig:postfit_param_antelope} shows the post-fit values of the fit parameters and their uncertainties for the discovery (ANTELOPE-based) CR and VR. 
Figure~\ref{fig:bkgfit_data_crvr_antelope} shows the resulting functions and residuals with respect to the CR and VR data.
These results indicate good ability of the 5-parameter polynomial to also model the ANTELOPE selected region.
\begin{figure}[!htbp]
\centering
   \includegraphics[width=0.65\textwidth]{figures/stats/postfit_param_antelope}
    \caption{Post-fit parameters for the ANTELOPE CR and VR.
    \label{fig:postfit_param_antelope}}
\end{figure}
\begin{figure}[!htbp]
\centering
   \includegraphics[width=0.95\textwidth]{figures/stats/bkgfit_data_crvr_antelope}
    \caption{Post-fit function and residuals for the ANTELOPE CR and VR.
    \label{fig:bkgfit_data_crvr_antelope}}
\end{figure}


%----------------------------------------------------------------------------------------
\subsubsection{Signal Mass Resolution \mt~Binning}
\label{subsec:binning}

In the discovery region, a binning for \mt~is determined that is based on the expected signal width. This is done to improved the BumpHunter performance.
The signal mass resolution for a given point is determined with a double-sided Crystal Ball fit to the mass. 
These fits are performed across Z' mass, and a linear fit to these values is performed to determine the optimal bin width across \mt.

The x-axis value used is a data-driven way to determine the appropriate value of \mt~for a given signal, given that the considerable \met~from the dark particles means that the truth Z' mass does not well approximate the peak \mt~value.
As the \met~in the final state means that the \mt~is always an underestimate of the Z' mass, the truth Z' mass can be used as an upper bound.
An integral is then performed backwards from that value until 60\% of the total signal yield is included. 
This window is referred to as the 60\% mass window; the mean of this window then provides an approximate localization of the signal mass peak in \mt.
Figure~\ref{fig:mass60percent_ex} shows some examples of this algorithm on several signal points of varying \rinv~and mass.
\begin{figure}[!htbp]
\centering
   \includegraphics[width=0.45\textwidth]{figures/stats/mass60percent_ex1}
   \includegraphics[width=0.45\textwidth]{figures/stats/mass60percent_ex2}
   \includegraphics[width=0.45\textwidth]{figures/stats/mass60percent_ex4}
   \includegraphics[width=0.45\textwidth]{figures/stats/mass60percent_ex3}
    \caption{Example determinations of the 60\% mass window means for several signal points.
    \label{fig:mass60percent_ex}}
\end{figure}

Figure~\ref{fig:linearfit_binning} shows the result of this linear fit for the four \rinv~values considered in the signal grid.
As expected, the resolution is considerably different for low and high \rinv~points.
\begin{figure}[!htbp]
\centering
   \includegraphics[width=0.52\textwidth]{figures/stats/linearfit_binning}
    \caption{Signal mass resolution for \mt~binning for the signal grid in (\rinv, mass) space.
    \label{fig:linearfit_binning}}
\end{figure}

A single \mt~binning for the final SR plotting and BumpHunting is determined by selecting a harmonized binning at low \mt~, and moving to wider bins at high \mt.
As for higher \rinv~signal points the mass resolution linear fit gives negative results, we require each bin to have a width of at least 100 GeV.
Figure~\ref{fig:bins_rinv} shows the resulting bins for each \rinv~category that comes from the mass resolution fits, with the addition of the minimum 100 GeV bin width requirement.
\begin{figure}[!htbp]
\centering
   \includegraphics[width=0.82\textwidth]{figures/stats/bins_rinv}
    \caption{\mt~bins based on the signal mass resolution and the minimum 100 GeV width requirement, for each \rinv~signal category.
    \label{fig:bins_rinv}}
\end{figure}

In order to have a final \mt~binning that is not highly model-dependent, we consolidate these four different bins into a single binning which is provided below:

%\textbf{[1500, 1600, 1700, 1800, 1904, 2029, 2177, 2356, 2569, 2824, 3128, 3493, 3929, 4451, 5075, 6000]}
\textbf{[1500, 1600, 1700, 1800, 1900, 2025, 2175, 2350, 2575, 2825, 3125, 3500, 3925, 4450, 5075, 6000]}


%----------------------------------------------------------------------------------------
\subsubsection{BumpHunter Fits}
\label{subsec:bhfits}

Figure~\ref{fig:antelope_bh_crvr} shows the result of running BumpHunter over the CR and VR \mt~spectra, binned according to the signal mass resolution binning described above. 
The background estimation here is derived by fitting the ANTELOPE regions with the polynomial fit function, and sampling from this function to create a binned histogram.
We define a spurious signal as any BumpHunter fit finding an excess with p-value $<$ 0.01 (taken from other dijet analysis eg. Ref.~\cite{ATLAS:2023azi}).
No spurious signal is observed, and p-values indicate good agreement with the background estimation.
\begin{figure}[!htbp]
\centering
   \includegraphics[width=0.95\textwidth]{figures/stats/antelope_bh_cr}
   \includegraphics[width=0.95\textwidth]{figures/stats/antelope_bh_vr}
    \caption{BumpHunter fits on the ANTELOPE \mt~spectra for both the CR and VR. In a signal-depleted region, good agreement with the background estimation is observed.
    \label{fig:antelope_bh_crvr}}
\end{figure}

Figure~\ref{fig:bh_asimov_pvals} shows BumpHunter p-values over 100 Asimov trials
The same fit success criteria as the PFN region is applied: p-value  $>$ 0.001 and fit status succeeding (0 or 1).
In both cases, no spurious signals are found (p-value $<$ 0.01).
\begin{figure}[!htbp]
\centering
   \includegraphics[width=0.45\textwidth]{figures/stats/bh_asimov_pvals_cr}
   \includegraphics[width=0.45\textwidth]{figures/stats/bh_asimov_pvals_vr}
    \caption{BumpHunter p-values extracted for 100 Asimov toys for both the ANTELOPE CR (top) and VR (bottom) showing the highest (left) and lowest (right) p-value fits. The number of events in the histogram deviates from 100 based on failed background-only fits.
    \label{fig:bh_asimov_pvals}}
\end{figure}





