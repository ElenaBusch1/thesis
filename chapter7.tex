\chapter{Machine Learning Tools}
\label{ch:ml_tools}

\section{Introduction}
The search for semi-visible jets presents an opportunity to use novel machine learning (ML) tools to uncover patterns in the behavior of dark QCD. The subtlety of the shower differences between dark QCD (signal) and SM QCD (background) motivates a complex model that can accept high-dimensional low-level information, such as travel level information, to best understand key differences between signal and background correlations. Additionally, the large number of theory parameters which can be chosen arbitrarily and affect the shape of the dark QCD shower motivate exploring a data-driven machine learning approach, which could be sensitive to a wider variety of dark QCD behavior. \par

To this end, two machine learning approaches are developed for this search, which are used in tandem. The first is a supervised ML method where the ML algorithm is built to maximize exclusion sensitivity to the specific generated SVJ signal models used in this analysis. Here, supervised refers to the use of full and correct \textit{labels}\footnote{In machine learning a label refers to the correct identification information for an input. In the case of the binary classifier algorithm discussed here, the label is either ``signal" or ``background".} for all events considered during model training, which necessitates training over simulated data. The second is a semi-supervised method, where training of the model is data-driven and labels are only partially provided during training. The semi-supervised ML algorithm broadens the discovery sensitivity of the search, and reduces the dependence on the exact theory parameters chosen for signal model simulation. \par

The two different ML algorithms used in this approach will be explained in the following sections, along with their application in the SVJ analysis strategy.

%------------------------------------------------
\section{Particle Flow Network (Supervised)}
\label{subsec:supervised}
The supervised machine learning approach maximizes discovery sensitivity for the SVJ signals considered in this thesis.
The networks learns the features of the SVJ signals, allowing the network to be highly efficient in selecting events that resemble the SVJ signal.

\subsection{Architecture Fundamentals}

A Particle Flow Network (PFN)~\cite{pfn} architecture is selected for two reasons: \textit{permutation invariant input modeling} to best describe the events consisting of an unordered set of particles, and a \textit{low-level input modeling} to take advantage of the ability of neural networks to uncover patterns in high-dimensional data. \textit{Low-level} refers to using detector level information such as individual particle tracks, rather than \textit{high-level} information such as reconstructed jet objects. Low-level inputs are generally high-dimensional; for instance, an event may have only 2 jets (dim-2), but each jet consists of 70 particles (dim-140). Low-level input modeling is chosen to capture the intricacies of dark QCD showers with may not express themselves in high level objects, as explored in Ref.~\cite{darkqcd}. Permutation invariant input modeling is chosen as the most accurate representation of a set of particles. In previous work such as Ref.~\cite{vrnn}, ordered input modeling has been observed to \textit{bias} the performance of low-level modeling tools. In this case bias means that the performance of the tool was observed to change substantially depending on the input ordering; however, there is no physics motivation for choosing any particular order. 

The input to the PFN is a collection of particles and their associated physics information, such as momentum and trajectory. Constructing the PFN involves the creation of new basis variables $\Phi$ for each particle in the input event. This transformation is summarized as $\vec{p_i} \rightarrow \vec{\Phi_i}$ where $\vec{p_i}$ is the physics information for the $i$th particle in the event, and $\vec{\Phi_i}$ is that same information encoded into the $\Phi$ basis. Permutation invariance is enforced by summing over the $\Phi$ basis for every particle in the event to create a new permutation invariant event representation $\mathcal{O}$. The creation of $\mathcal{O}$ from $M$ particles $\vec{p}$ with $d$ physics features each can be expressed as:

\begin{equation}
  \mathcal{O}(\{\vec{p_1},...,\vec{p_M}\}) = \sum_{i=1}^M \Phi_i(\vec{p_i})
  \label{eq:pfn}
\end{equation}

where $\Phi : \mathbb{R}^d \rightarrow \mathbb{R}^l$ is a per particle mapping, with $l$ being the dimension of the new basis $\mathcal{O}$. Figure~\ref{fig:pfn_paper} gives a graphical representation of the use of summation in the PFN over per-particle information to create a permutation-invariant event representation. \par
\begin{figure}[!htbp]
\centering
   \includegraphics[width=0.7\textwidth]{figures/ml/pfn_paper}
    \caption{The Energy/Particle Flow Network concept, from Ref.~\cite{pfn}. The physics input information is represented as arrows on the left, for an arbitrary number of particles. The $\Phi$ transformation converts these arrows to 3 graphs, indicating the $\Phi$ basis dimension $l$ is 3 in this example. The graphs are then summed for all particles to create $\mathcal{O}$, or the event representation.
    \label{fig:pfn_paper}}
\end{figure}

The $\Phi$ basis transformation is implemented via a deep neural network. The output of the neural network is summed as indicated in Equation~\ref{eq:pfn} to create the new permutation invariant event representation $\mathcal{O}$. $\mathcal{O}$ then becomes the input of a second deep neural network $F$. $F$ is a classifier network which separates signal and background events. Figure~\ref{fig:pfn_arch} provides an annotated diagram of the PFN architecture as used in this analysis. 
\begin{figure}[!htbp]
\centering
   \includegraphics[width=0.8\textwidth]{figures/ml/pfn_arch}
    \caption{An annotated diagram of the PFN architecture~\cite{pfn}. $y$ and $\phi$ represent geometric trajectory information for the input particles, $z$ represents energy information, and PID encompasses any other particle ID information in the input. PID is presented in the diagram as a 1-dimensional input, but could represent multiple input dimensions.
        \label{fig:pfn_arch}}
\end{figure}

%--------------------
\subsection{Input Modeling, Scaling, and Rotation}
\label{sec:input_model}
In this implementation, the particle input information comes from all tracks associated to the leading and subleading jets. The track association method is Ghost association, as discussed in Section~\ref{sec:ghost}. A single jet tagger strategy was also considered, but utilizing tracks from both leading jets creates a more complete low-level picture of the event. The choice of the two leading jets is justified in Chapter~\ref{ch:analysis}. If we consider the dijet topology of semi-visible jets as illustrated in Figure~\ref{fig:svj_pic}, the advantage of modeling both leading jets simultaneously becomes clear. In the semi-visible jet model presented in Ref.~\cite{darkqcd}, \met~in the event is expected to arise due to an imbalance in the number of visible tracks of the two jets associated to the dark quark decay.\par

\begin{figure}[!htbp]
\centering
   \includegraphics[width=0.4\textwidth]{figures/ml/dijet_topology}
    \caption{An illustration of the expected dijet behavior of semi-visible jets, where one jet is closely aligned with \met (MET). In the figure two jet cones $j_1$ and $j_2$ are illustrated, along with their associated momentum vectors $\vec{p_1}$ and $\vec{p_2}$. 
        \label{fig:svj_pic}}
\end{figure}

Each track is described using six variables: the four-vector of the track (\pt, $\eta$, $\phi$, E), and the track displacement parameters $d_0$ and $z_0$, where $d_0$ measures displacement in the radial direction from the beamline and $z_0$ measures displacement along the beamline from the primary interaction point. Figure~\ref{fig:trackcoordinates} illustrates these coordinates. Up to 80 tracks per jet are allowed, which is a threshold chosen to generally include all the tracks in the jet, which leads to maximal performance.\par %Figure~\ref{fig:ntracks} shows the track multiplicity in the leading and subleading jet for the signal and background samples used in training. \par

\begin{figure}[!htbp]
\centering
   \includegraphics[width=0.4\textwidth]{figures/ml/trackcoordinates}
    \caption{Illustration of track coordinates $d_0$ and $z_0$.
    \label{fig:trackcoordinates}}
\end{figure}

%\begin{figure}[!htbp]
%\centering
 %  \includegraphics[width=0.95\textwidth]{figures/ml/ntracks}
 %   \caption{Distributions of the track multiplicity in the leading and subleading jets, comparing signal and background PFN training samples.
%    \label{fig:ntracks}}
%\end{figure}

These tracks (up to 160 total) are the input to the PFN. Referencing Equation~\ref{eq:pfn}, this corresponds to $M = 160$ (number of particles) and $d = 6$ (number of features per particle). The two leading jets and their associated tracks are rotated so that the vector sum of the jets, or system average, is aligned with $(\eta,\phi) = (0,0)$. The rotation can be summarized as 
\begin{subequations}
    \begin{align}
       \eta_{i}' &= \eta_i - \bar{\eta},  \\
        \phi_{i}' &= \phi_i - \bar{\phi}
    \end{align}
\end{subequations}
where ($\bar{\eta}, \bar{\phi}$) is the average angle of the dijet system,  ($\eta_{i}, \phi_{i}$) are the original track coordinates, and ($\eta_{i}', \phi_{i}'$) are the rotated track coordinates. Figure~\ref{fig:jet_rotate} illustrates the rotation process. The rotation ensures that the information used by the algorithm is the relative orientation of the jets (and associated tracks) to each other, not their absolute position in the detector. Each track is normalized to its relative fraction of the total dijet system energy and transverse momentum; this enforces agnosticism to the total energy and transverse momentum of the event. The rotation and scaling are motivated by the procedures described in Ref.~\cite{pfn} to improve the performance of the PFN. 

\begin{figure}[!htbp]
\centering
   \includegraphics[width=0.65\textwidth]{figures/ml/jet_rotate}
    \caption{A diagram demonstrating how the two jet system is rotated in $(\eta,\phi)$. The jet cones and associated jet tracks are illustrated. The dashed tracks represent dark hadrons while the solid tracks represent SM hadrons. The system average $(\bar{\eta},\bar{\phi})$ is shown in red and an example track with coordinates $(\eta_i,\phi_i)$ is shown in purple.
    \label{fig:jet_rotate}}
\end{figure}

Finally, each of the 6 track variables is scaled so that its range is [0,1]. This is a common preprocessing step that ensures the input data is bounded over a similar range, so that arbitrarily large values don't develop an outsized impact on the model. The track momentum and energy normalization mentioned above naturally enforces that these values are restrained between [0,1]. The $\eta$ and $\phi$ values are naturally bounded, so for these values the $\eta$ tracking range\footnote{This range is dictated by the $|\eta|$coverage range of the Inner Detector, as shown in Table~\ref{tab:atlas_requirements}} of [-2.5, 2,5] and the full $\phi$ range [$-\pi$, $\pi$] are mapped to [0,1]. The displacement variables are restricted to [0,1] via the standard \textsc{MinMaxScaler}~\cite{scikit-learn} method which determines the minimum and maximum values observed in training, and maps those boundaries to 0 and 1 respectively. \par

Figure~\ref{fig:pfn_datamc_input} illustrates that the data is well modeled by the MC at track level. Figure~\ref{fig:pfn_bkgsig_input_kin} shows the kinematics of each of 6 track variables for background and signal. Figure~\ref{fig:pfn_bkgsig_input_rot} shows each of the 6 track variables after scaling and rotation have been applied, demonstrating the impact of these procedures, as well as the track level similarities and differences between the background SM QCD processes and the signal SVJ processes. \par

The $\phi$ distribution is of note for its jagged appearance in QCD MC. This arises due to dead tile calorimeter cells in certain $\phi$ regions, the effects of which are seen in data and modeled in QCD MC but not modeled in SVJ signal MC. Appendix~\ref{subsec:tileCal} contains more information about how the issue was addressed in data. The distribution is not of concern for the PFN training because of the rotation process, which replaces the information about absolute detector $\phi$ measurements with the relative $\phi'$ measurement. This is illustrated in Figure~\ref{fig:pfn_bkgsig_input_rot}, where it is observed that for both signal and background the tracks are clustered back to back, centered at $-\pi$/2 and $\pi$/2 (0.25 and 0.75 after scaling). The only remaining difference is that the signal tracks are more likely to be close to the system average $\bar{\phi}$ than the background jet tracks. This is demonstrated by the excess of signal events in the center of the $\phi'$ plot. This orientation difference is a real feature of the signal model, confirmed in Figure~\ref{fig:presel_vars2} which illustrates that signal jets are more likely to have low $\Delta\phi$ than background jets. 

\begin{figure}[!htbp]
   \centering
   \includegraphics[width=0.99\textwidth]{figures/ml/pfn_datamc_input}
    \caption{The 6 PFN track variables in background MC (blue) and data (orange), after the scaling and rotation procedure is applied. There is excellent modeling of the data by the MC within the track variables. The slight discrepancy in the $\phi$ distribution due to the inaccuracies of modeling dead TileCal cells in the QCD MC is considered. The level of discrepancy is determined to be within tolerance given that the final result with be data driven and the QCD model is used in the PFN training only.
    \label{fig:pfn_datamc_input}}
\end{figure}

\begin{figure}[!htbp]
    \centering
     \includegraphics[width=0.99\textwidth]{figures/ml/pfn_bkgsig_input_kin}
     \caption{The 6 PFN track variables in background MC (blue) and signal MC (orange) before scaling and rotation. The track kinematics are largely similar, and the variation in the $\phi$ distribution is explained in the text.}
      \label{fig:pfn_bkgsig_input_kin}
\end{figure}

\begin{figure}[!htbp]
    \centering
    \includegraphics[width=0.99\textwidth]{figures/ml/pfn_bkgsig_input_rot}
     \caption{The 6 PFN track variables in background MC (blue) and signal MC (orange) after scaling and rotation. The $\phi$ distribution is modified by the rotation procedure, as explained in the text.}
     \label{fig:pfn_bkgsig_input_rot}
\end{figure}

\clearpage

%--------------------
\subsection{Training}
\label{sec:pfn_training}

As seen in Figure~\ref{fig:pfn_arch}, two networks are defined and combined for the PFN architecture. In our implementation the input layer has a dimension of 6, accounting for the 6 track variables described in the previous section. The first network, termed the $\Phi$ network, creates the per-particle set representation as illustrated in Figure~\ref{fig:pfn_paper}. The $\Phi$ network has 2 hidden layers each of dimension 75, and an output later of dimension 64. These dimensions were chosen via an optimization procedure which balanced network complexity (achieved with more dimensions) against training time (achieved with fewer dimensions). The two hidden layers and $\Phi$ output layer all use a \textsc{relu} activation function~\cite{scikit-learn}, following the work of Ref.~\cite{pfn}. 

The input layer of the classifier $F$ network is required to have the same dimension as the output layer of the $\Phi$ network, and therefore takes dimension 64. This network contains 3 hidden layers with 75 nodes each, and again uses \textsc{relu} activation~\cite{scikit-learn}. The final layer is the binary classifier result with dimension 2, which uses a \textsc{softmax} activation~\cite{scikit-learn} that is well suited for classification. The loss function for the complete PFN network is \textsc{CategoricalCrossentropy}~\cite{scikit-learn}, which is a standard loss function for DNN classifiers. The standard Adam optimizer~\cite{adam}~\cite{scikit-learn} is used with a learning rate of 0.001. The learning rate was reduced from the nominal learning rate of 0.01 presented in Ref.~\cite{pfn} to prevent overtraining.\par

The PFN is a supervised algorithm, and is therefore trained on a labeled mixture of signal and background events. The signal input is an even mixture of all signal points considered in this analysis. Although the full simulated background for this analysis is composed of several SM processes as discussed in Section~\ref{subsec:bkg_mc}, QCD is the dominant background. Training with a QCD-only background sample is determined to produce better results than training using the full background mixture. Including MC backgrounds that are enriched in \met~(recall Figure~\ref{fig:bkg_mc}) reduces the ability of the PFN to classify SVJ signals. This is illustrated in the comparison of output classifier distributions in Figure~\ref{fig:pfn_MC_training_mixture}. The signals used for training are the same in both cases. When training with a QCD-only background, high \met~data and MC is more likely to be classified as signal like; however the increased signal performance means that overall \textit{sensitivity}\footnote{Sensitivity is a measure of the ability of an analysis to detect the signal, discussed further in Section~\ref{sec:eventsel}} is higher with a QCD-only training. Additional studies on the optimal PFN training event mixture are available in Appendix~\ref{app:pfn_qp}. \par

\begin{figure}[!htbp]
\centering
   \includegraphics[width=0.98\textwidth]{figures/ml/pfn_MC_training_mixture}
    \caption{PFN score for full-background MC (black), data (red), and 2 representative signal points (green). The left plot is from a QCD-only training, while the right plot is from a full-background training. The histograms have been normalized to visualize the shapes better - the actual number of plotted events is shown in the legend. In the left plot we observe that both signal points are strongly classified as signal-like. In the right plot we observe less background contamination in the high score region, but worse signal classification. Both PFN trainings were tested for their effect on the analysis sensitivity and the QCD-only training was found to be favorable. 
    \label{fig:pfn_MC_training_mixture}}
\end{figure}

500k QCD MC background events and 500k SVJ signal events are used to train the network. The network is trained for 100 epochs. 20\% of the training events are used for training validation. Figure~\ref{fig:pfn_loss} shows the loss during training, which is stable and shows no indication of overtraining, and the final score that provides signal-background discrimination.

\begin{figure}[!htbp]
\centering
   \includegraphics[width=0.9\textwidth]{figures/ml/pfn_loss_score}    
    \caption{PFN architecture loss during training as a function of epoch (left) and the evaluated score for signal and background training samples (right). The loss vs. epoch plot shows that the network is not overtrained. The score plot shows a good separation between signal and background.
    \label{fig:pfn_loss}}
\end{figure}

Optimization studies were performed on the PFN, varying the number of training epochs, number of training events, learning rate, number of nodes, and dimension of the $\Phi$ basis. A summary of these studies is presented in Appendix~\ref{app:pfn_qp}. The model presented here represents an optimal choice across these parameters.

%--------------------
\subsection{Performance}
\label{sec:pfn_performance}

The performance of the PFN is assessed via the AUC for each SVJ signal point.
Although the PFN is trained against QCD MC only, the performance is evaluated using data as the background sample, since the ultimate task of the PFN is to separate SVJ signals from data, which is dominated by SM processes.

Figure~\ref{fig:pfn_roc} shows the ROC curve of one such signal point, illustrating a smooth response.
Figure~\ref{fig:pfn_AUC_score_grid} shows the AUC of the PFN across the SVJ signal grid, demonstrating that AUC $>0.5$ is satisfied for all SVJ signals.

\begin{figure}[!htbp]
\centering
   \includegraphics[width=0.7\textwidth]{figures/ml/pfn_roc}
    \caption{ROC for the PFN, using SVJ signal events (true positive) and data (false positive).
    \label{fig:pfn_roc}}
\end{figure}

\begin{figure}[!htbp]
\centering
   \includegraphics[width=0.7\textwidth]{figures/ml/pfn_AUC_grid}
    \caption{AUC for the PFN, shown for each signal in the SVJ grid.
    \label{fig:pfn_AUC_score_grid}}
\end{figure}

Figure~\ref{fig:pfn_score_all} shows the output score distribution for data and four signals, illustrating the range of scores received by data events in comparison to signal events.
As expected, most data events receive a background-like score (close to 0.0), indicating that the data is dominated by SM processes consistent with the background.
Most signal events receive a signal-like score (close to 1.0).
An optimization procedure determined that a selection of \textbf{PFN score > 0.6} can improve signal sensitivity across the grid.
The optimization procedure considered the cut that would maximize sensitivity as measured by $s/\sqrt{b}$, where $s$ the number of signal events accepted and $b$ is the number of background events selected.
This score selection is incorporated into the analysis design described in Chapter~\ref{ch:analysis}. 

\begin{figure}[!htbp]
\centering
   \includegraphics[width=0.5\textwidth]{figures/ml/pfn_score_all}
   \caption{Illustration of the PFN score selection, showing the separation between data (black) and 4 signal points (blue and green). The legend information takes the form ``$m_{Z'}$ \rinv'' for the signal. The PFN score selection value is shown by the pink line. Only events with a score > 0.6 will be accepted for use in the analysis. We see that most background (data) is rejected, while most signal is accepted.}
   \label{fig:pfn_score_all}
\end{figure}

%The agreement between data and background MC is illustrated in Figure~\ref{fig:mlscore_effComp}. The agreement is generally good, although some slope is observed in the ratio between the two shapes. The data has a small bias towards higher PFN scores compared to the background MC. However, the PFN score is only used in the analysis to make a selection on data events (PFN score > 0.6). The difference in selection efficiency for data and background MC <5.0\%, which is negligible for this analysis. 
%\begin{figure}[!htbp]
%\centering
%   \includegraphics[width=0.5\textwidth]{figures/ml/mlscore_effComp}
%    \caption{PFN score comparison between normalized data and background MC shapes. Some slope is observed in the ratio panel.
%    \label{fig:mlscore_effComp}}
%\end{figure}

\clearpage


\subsection{ANTELOPE (Semi-supervised)}
\label{subsec:unsupervised}

The semi-supervised analysis approach broadens the discovery sensitivity of the search through the use of semi-supervised ML, where training of the model is data-driven and labels are only partially provided during training.
While broad sensitivity is a general key goal of LHC searches, it is particularly motivated in the case of dark QCD models, which can lead to widely varying topologies depending on the values of model parameters.
In the case of SVJs, the \rinv~fraction in the jet can dramatically vary the \met, shower shape, and other key features, making it difficult to find a single standard analysis variable that can distinguish all signal topologies from QCD.

%--------------------
\subsubsection{Architecture Fundamentals}
The model-independent search region of this analysis is implemented with a novel ML approach that builds on the ANTELOPE architecture to construct a tool that is capable of performing low-level anomaly detection with permutation-invariant inputs.
This tool, referred to as \textbf{ANomaly deTEction on particLe flOw latent sPacE (ANTELOPE)}, is a custom solution designed for this analysis.

ANTELOPE uses the supervised signal vs. background training of the PFN network described in the previous section to generate a permutation invariant latent space that is representative of the original input variables, encodes the input events into these latent space variables $\mathcal{O}$, and trains a \textit{variational autoencoder} (VAE) over the events modeled as PFN latent space variables. A VAE is a common architecture used for anomaly detection and data-driven ML training. It has been used in previous ATLAS searched to model jet level information, such as the search presented in \cite{yxh} using the recurrent architecture described in \cite{vrnn}. One of the limitations of a recurrent architecture is the need to order the low level inputs, which affects the performance of the tool. Jet constituent information is intrinsically unordered, and therefore a permutation invariant approach removes this element of arbitrary decision making from the modeling process. A visual example of the ANTELOPE inputs is given in Figure~\ref{fig:antelope_input_rep}. 

\begin{figure}[!htbp]
\centering
   \includegraphics[width=0.9\textwidth]{figures/ml/antelope_input_rep}
    \caption{A visual representation of the 64 PFN latent space variables which create the input of the VAE component of ANTELOPE. The left shows a 2D histogram of the PFN latent space index (0-63) versus the value assumed by that index. The right shows 1D histograms of two particular PFN latent space variables. 
    \label{fig:antelope_input_rep}}
\end{figure}

The input to the model is the same 6 track variables for the leading 160 tracks of the leading two jets, as presented in Section~\ref{sec:input_model}. The track information is encoded to the PFN $\Phi$ latent space using the pre-trained $\Phi$ network (trained according to the steps outline in Section~\ref{sec:pfn_training}. The resulting $\Phi$ basis is summed to created the fixed length symmetric representation $\mathcal{O}$. The VAE is then trained in an unsupervised way using inputs encoded to $\mathcal{O}$ from data events only. The VAE is given no knowledge of the signal model during training. 
It is able to perform anomaly detection through an encoder stage which does a lossy compression on the input to a lower-dimensional latent space, and a decoder stage that samples from that latent space and generates an output of the original dimensionality.
By using the reconstruction error as a loss, this process enables the VAE to develop a knowledge of the underlying data structure, thereby isolating new out-of-distribution events by their high reconstruction error. 
This strategy is semi-supervised because the tool has some knowledge of correct labels (eg. through the PFN latent space embedding) but is followed by a data-driven unsupervised stage. 

Figure~\ref{fig:antelope_arch} provides a diagram of the ANTELOPE architecture.
\begin{figure}[!htbp]
\centering
   \includegraphics[width=0.9\textwidth]{figures/ml/antelope_arch}
    \caption{An annotated diagram of the ANTELOPE architecture.
    \label{fig:antelope_arch}}
\end{figure}


%--------------------
\subsubsection{Training}

The VAE stage of the ANTELOPE network is trained directly over a subset of data events at preselection (6.7 million available, 500,000 used, with a 80\% / 20\% training/test split).
The input dimensionality of the VAE has to match the encoded $\Phi$ dimension of the PFN, in this case 64. 
The encoder has an encoding layer that brings the dimensionality to 32, and a final layer that compresses to the latent space dimension of 12. 
The network is trained for 50 epochs, with a learning rate of 0.00001.  
The loss $\mathcal{L}$ is the sum of two terms, the mean-squared error (MSE) of input-output reconstruction, and the Kullback-Leibler divergence (KLD).

\begin{equation}
\label{eq:vrnnloss}
\mathcal{L} = \sum_i L_i = \sum_i | \Phi_i^2 - \Phi\prime_i |^2 + \lambda D_{\text{KL}}
\end{equation}

As the PFN inputs are sufficiently normalized to remove any spurious information from training, no additional normalization is applied to the PFN encoded inputs.
The final ANTELOPE score used in the analysis is produced by applying a log + sigmoid transformation function to the total evaluated loss $\mathcal{L}$. 

Figure~\ref{fig:antelope_loss} shows the loss during training.
\begin{figure}[!htbp]
\centering
   \includegraphics[width=0.5\textwidth]{figures/ml/antelope_loss}    
    \caption{ANTELOPE architecture loss during training as a function of epoch.
    \label{fig:antelope_loss}}
\end{figure}


%--------------------
\subsubsection{Performance}
\label{subsec:antelope_perf}

As with the PFN, the ANTELOPE performance is assessed via the area-under-curve (AUC) of the receiver operating characteristic (ROC) associated to evaluating the ANTELOPE on the test set of signal and background events.
Figure~\ref{fig:antelope_score} shows the output score distribution in data and total background MC, showing a very flat ratio and motivating the use of MC for studies of the ANTELOPE score.

\begin{figure}[!htbp]
\centering
   \includegraphics[width=0.48\textwidth]{figures/ml/antelope_score.pdf}
   \includegraphics[width=0.48\textwidth]{figures/ml/antelope_score_mcsig}
    \caption{ANTELOPE score distribution comparing data and the total background MC (left), with good agreement observed between data and simulated background, and comparing all background MC to signals (right), revealing good discrimination power.
    \label{fig:antelope_score}}
\end{figure}

Figure~\ref{fig:antelope_AUC_score_grid} shows the AUC of the ANTELOPE across the SVJ signal grid, demonstrating strong discrimination capability even in the varying corners of phase space. 
Compared to the supervised PFN method, the ANTELOPE is not as performant (as expected due to the absence of signal model in training).
However, a selection on events with high ANTELOPE score nonetheless provides a 10-40\% increase in signal significance by removing background and isolating the long tail of anomalous events.

\begin{figure}[!htbp]
\centering
   \includegraphics[width=0.7\textwidth]{figures/ml/antelope_AUC_score_grid}
    \caption{AUC from the ANTELOPE score for each signal in the SVJ grid.
    \label{fig:antelope_AUC_score_grid}}
\end{figure}

\paragraph{Model Independence} 

The unsupervised component of training the ANTELOPE network is expected to give it a more generalized sensitivity to new physics with \met~and jet activity, beyond the scope of the SVJ grid. 
To assess this, alternative signal models are evaluated with the trained ANTELOPE network, as optimized for the SVJ grid, and their sensitivity in the analysis selection is evaluated.

The following alternate signal models were considered: 
\begin{itemize}
\item Z' $\rightarrow$ $t\bar{t}$ 
\item W' $\rightarrow$ WZ 
\item Gluino pair production $\rightarrow$ R-hadron + LSP (\met) with gluino masses 2000/3000 GeV, LSP mass 100 GeV, and lifetime 0.03 ns (LSP = \textit{lightest supersymmetric particle})
\item Emerging jets s-channel with mass 1000 GeV and lifetime 1ns 
\end{itemize}

Figure~\ref{fig:antelope_altsig} shows the distribution of these signals in the PFN score and the ANTELOPE score.
This comparison reveals that ANTELOPE is sensitive to \met~in the event; it classifies signals with no real \met~, like the all-hadronic Z' and W' decays (given our imposed lepton veto) as data-like, but the distributions for signals with \met~such as SVJs, R-hadrons, and emerging jets have distributions with higher anomaly score tails.
\begin{figure}[!htbp]
\centering
   \includegraphics[width=0.48\textwidth]{figures/ml/pfn_altsignals}
   \includegraphics[width=0.48\textwidth]{figures/ml/antelope_altsignals}
    \caption{Comparing data and the alternate signal models for the PFN score (left) and ANTELOPE score (right). The emerging jet signal is an example of the gain of the model-independent ANTELOPE approach, where it has a bimodal shape in PFN score but is clearly tagged as anomalous by ANTELOPE.
    \label{fig:antelope_altsig}}
\end{figure}

Figure~\ref{fig:antelope_pfn_comp} shows a comparison of the sensitivity of the PFN and ANTELOPE regions across a variety of signals, including the combined SVJ signal used to train the PFN.
The benefit of the unsupervised stage of ANTELOPE in enhancing model independence is clearly seen through the boost in performance for other signal models, namely the gluino and emerging jet signals, which have more \met~than the W' and Z' signals (all-hadronic) that were also tested. 
As commented above, the PFN outperforms ANTELOPE as expected, because it was designed explicitly for the task of classifying SVJs from background, demonstrating the power of supervised learning for the model-specific approach.

\begin{figure}[!htbp]
\centering
   \includegraphics[width=0.95\textwidth]{figures/ml/antelope_pfn_comp}
    \caption{Comparing data and the alternate signal models in terms of sensitivity (S/$\sqrt{B}$) for the PFN and ANTELOPE tools, applying the selection that is used in the analysis. The ANTELOPE network is found to provide significant added sensitivity to alternate signals such as the gluino$\rightarrow$ R-hadron and emerging jets, which have higher \met~than the SVJs.
    \label{fig:antelope_pfn_comp}}
\end{figure}


Studies on the ANTELOPE architecture and comparisons to other methods can be found in Appendix~\ref{app:aevsantelope}.





