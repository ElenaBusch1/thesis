\chapter{Results}
\label{ch:results}
The final results of this analysis are the polynomial fit to the \mt~distribution in the SVJ Fit SR, and the BumpHunter evaluation of the \mt~distribution in the Discovery SR. In the SVJ Fit region, systematic uncertainties are evaluated on the signal model, and \textit{limits}\footnote{A limit is an upper bound of the branching ratio of a signal process} on the observed $Z'$ production cross-section are set. 

\section{SVJ Fit Result}
\label{sec:results_svj}
Figure~\ref{fig:unblinded_PFN_bonly} shows the unblinded \mt~spectrum in the SVJ Fit SR with a background-only fit. 
The fit is successful and has a p-value of 0.265, indicating the data is compatible with the background hypothesis. 
Table~\ref{tab:unblinded_params} gives the values and uncertainties for the five parameters of the polynomial fit.
\begin{figure}[!htbp]
\centering
   \includegraphics[width=0.5\textwidth]{figures/results/unblinded_PFN_bonly}
    \caption{\mt~in the unblinded SVJ Fit SR with a background-only fit (p-value = 0.265).
    \label{fig:unblinded_PFN_bonly}}
\end{figure}

\begin{table}[!htbp]
\centering
   \includegraphics[width=0.45\textwidth]{figures/results/postfit_param_pfnSR}
    \caption{Post-fit parameters for the PFN SR. $p1$ can also be considered $N_{bkg}$ or the normalization factor.
    \label{tab:unblinded_params}}
\end{table}

%Figure~\ref{fig:unblinded_limits_nosyst} shows the expected and observed limits in the unblinded SR, without signal systematics considered in the fit.
%As the data was found in the b-only fit to be compatible with background, good agreement of observed and expected limits is seen.
%\begin{figure}[!htbp]
%\centering
 %  \includegraphics[width=0.95\textwidth]{figures/results/unblinded_limits_nosyst}
 %   \caption{Expected and observed 95\% CL limits in the unblinded SR, as a function of Z' masses for \rinv=0.2 (top left), 0.4 (top right), 0.6 (bottom left), 0.8 (bottom right); no systematics.
%    \label{fig:unblinded_limits_nosyst}}
%\end{figure}

\subsection{Systematics}
As is typically done in dijet resonance searches using a polynomial fit \cite{dijet_uncert}, the systematic uncertainties in this analysis are applied only to the signal and not to the background.
This is because the background expectation is determined entirely from the data in the SR via the polynomial fit.
Therefore the only uncertainty on the background is the statistical uncertainty, which is reflected in the uncertainty associated to each of the five freely floating parameters determined in the fit.

A variety of systematics on the signal shape and yield are considered.
The most significant of these is the \textit{spurious signal} systematic, which quantifies the level of signal observed in the absence of signal injection.
Experimental uncertainties on the luminosity and jet reconstruction are studied.
Finally, uncertainties on the MC simulation of the SVJ theory model are also considered.

%------------------------------------------------------------
\subsubsection{Spurious Signal}

The spurious signal uncertainty is assessed following the prescription in Ref.\cite{smooth_bkg}.
In this procedure, the spurious signal is defined using pseudo-data experiments, which are drawn from a smoothed template as described in Section~\ref{subsec:fit_bkgonly}.
A spurious signal uncertainty is included in the fit as a \textit{yield} uncertainty on each signal point.

The spurious signal $\mu_{\text{spur}}$ is quantified for each signal as the mean number of signal events fitted across 100 signal-free pseudo-data experiments. 
To determine if the amount of spurious signal is tolerable, the threshold $\mu_{\text{spur}}$/$\sigma_{\text{spur}} < 0.5$ is used \cite{smooth_bkg}.
$\sigma_{\text{spur}}$ for each signal point is the standard deviation on the number of fitted signal events across the 100 pseudo-data experiments.
Figure~\ref{fig:spursig_ex} gives examples of these pseudo-data experiments, revealing Gaussian distributions from which the mean and standard deviation used for this measurement are taken.
%Figure~\ref{fig:spursig_nevents} shows the determined spurious signal as a function of Z' resonance mass, for both low and high \rinv~points.

Figure~\ref{fig:spursig} shows the $\mu_{\text{spur}}$/$\sigma_{\text{spur}}$ metric.
The requirement for $\mu_{\text{spur}}$/$\sigma_{\text{spur}}$ < 0.5 is easily satisfied across the signal grid.
\begin{figure}[!htbp]
\centering
   \includegraphics[width=0.8\textwidth]{figures/systs/spursig_ex}
    \caption{Example spurious signal fits, indicating a Gaussian distribution around the mean of spurious signal events.
    \label{fig:spursig_ex}}
\end{figure}
\begin{figure}[!htbp]
\centering
   \includegraphics[width=0.6\textwidth]{figures/systs/spursig}
    \caption{Spurious signal as a function of resonance mass. The requirement $\mu/\sigma <0.5$ is satisfied for all signal points, where $\mu$ is the mean number of spurious signal events and $\sigma$ is the standard deviation of the number of spurious signal events from 100 pseudo-data experiments.
    \label{fig:spursig}}
\end{figure}


%As an additional verification of the size of the spurious signal, we perform the same check in the VR, as shown in Figure~\ref{fig:spursig_vs_mass_vr}.
%The size of the fitted spurious signal in the VR is consistent with that of the CR, indicating a good estimate of this uncertainty and extrapolation across regions.
%\begin{figure}[!htbp]
%\centering
%   \includegraphics[width=0.6\textwidth]{figures/systs/spursig_vs_mass_vr}
%    \caption{Fitted spurious signal in the VR, consistent with the CR and within the target size of $<$ 0.5 $\sigma$.
 %   \label{fig:spursig_vs_mass_vr}}
%\end{figure}

%------------------------------------------------------------
\subsubsection{Experimental Uncertainties}
The main experimental uncertainties are on the recorded luminosity, \textit{jet energy scale}, and \textit{jet energy resolution}.
The jet energy scale (JES) corrects for the non-compensating calorimeter response and jet energy losses in passive detector material \cite{jes_jer}.
The jet energy resolution (JER) applies a correction based on the ratio between a jet's true energy and its reconstructed energy, as determined in simulation.
Systematics uncertainties on the JES and JER processes must be considered for any analysis using reconstructed jets.

A flat yield uncertainty of 0.83\% is applied for all signals, corresponding to the uncertainty reported on the luminosity measurement by the LUCID detector \cite{lucid_uncertainty}. 

The JES and JER uncertainties are evaluated on each signal point for their impact on both the yield and shape of the \mt~distribution.
Table~\ref{tab:exp_syst} summarizes the range impact on the yield for each uncertainty.
The impact of these uncertainties on the signal yield is generally negligible in comparison to the spurious signal systematic, which ranges from 4.2\% in the case of the lowest $Z'$ mass points to >100\% in the case of the highest $Z'$ mass points.
In the 2000 GeV $Z'$ mass case (which has the lowest relative spurious signal uncertainty), the maximum yield difference due to experimental uncertainty is 0.53\%, or almost an order of magnitude reduced compared to the spurious signal uncertainty.

\begin{table}
\centering
  \begin{tabular}{ |c|c| }
    \hline
    Uncertainty & Effect on Yield [\%] \\
    \hline
     Luminosity & 0.83 \\
     JES & 0.04 - 1.39 \\
     JER & 0.01 - 0.64 \\
    \hline
  \end{tabular}
  \caption{Summary of Experimental Uncertainties and their impact on the yield of MC signal events.}
  \label{tab:exp_syst}
\end{table}

The impact of the JES and JER uncertainties on the shape of the \mt~distribution is also considered.
An example individual JES variation is shown in Figure~\ref{fig:jes_uncert}, illustrating the minimal impact of this uncertainty on the shape of \mt.
\begin{figure}[!htbp]
\centering
   \includegraphics[width=0.6\textwidth]{figures/results/jes}
    \caption{\mt~of the 3500 GeV $Z'$, \rinv~= 0.2 signal point, shown with an example JES variation. The nominal shape (``Nom''), 1$\sigma$ up (``Up''), and 1$\sigma$ down (``down") variations are shown. The variation is seen to have a negligible impact on the signal shape.
    \label{fig:jes_uncert}}
\end{figure}

To make a conservative estimate of their impact on the shape, all shape uncertainty sources are summed in quadrature, bin-by-bin.
This results in a maximum ``up'' variation and a maximum ``down'' variation.
The the impact of these maximal shape variations on the $Z'$ production cross-section limit is evaluated, and uncertainty on this limit is propagated to the final limit bands. 
The impact is generally seen to be quite small, changing the limit variation by 0.2 fb at most. 
An example of the variations summed in quadrature is shown in Figure~\ref{fig:jetcp_sumq}. 
\begin{figure}[!htbp]
\centering
   \includegraphics[width=0.62\textwidth]{figures/results/jetcp_sumq}
    \caption{\mt~of the 3500 GeV $Z'$, \rinv~= 0.2 signal point, shown with the sum in quadrature of all JES and JER variations. The nominal shape (``Nom"), 1$\sigma$ up (``Up"), and 1$\sigma$ down (``down") variations are shown.
    \label{fig:jetcp_sumq}}
\end{figure}

%------------------------------------------------------------
\subsubsection{Theory Uncertainty}
Uncertainty on the parameters of the signal model are also considered. 
The primary theory uncertainty source is the tuning of the parton shower in \textsc{Pythia8} \cite{parton_shower}. 
Jet structure and extra jet production within the event depend on the modeling of initial state radiation (ISR), final state radiation (FSR) and behavior of multiple parton interactions (MPI) within an event.
A variety of MC generation tuning parameters govern the behavior of ISR, FSR and MPI in the signal generation.
Ref~\cite{pythia8_tunes} describes how these parameters are condensed into 10 variations which capture the maximal range of impact for these tuning parameters. 

The 10 variations (representing 5 up/down variation pairs) are evaluated for the SVJ signal shapes. 
Figure~\ref{fig:isrfsr} provides a look at the effect of these variations on the SVJ \mt~signal shape. 

\begin{figure}[!htbp]
\centering
   \includegraphics[width=0.7\textwidth]{figures/systs/isrfsr}
    \caption{Signal distribution of \mt, varying the ISR, FSR and MPI configuration.
    \label{fig:isrfsr}}
\end{figure}

All 10 variations are determined to be flat within uncertainty, and thus the systematic is considered for its impact on the signal yield.
The variation in the signal yield is at most 5\% (TODO - determine exactly), which is incorporated into the fitted systematics. 
The spurious signal uncertainty is dominant for all but the lowest mass signal points.

%------------------------------------------------------------
\subsection{Interpretation}
Using a modified frequentist approach \cite{freq}, \textit{exclusion limits} at the 95\% confident level (CL) are derived.
Exclusion limits refer to determining the maximum (or \textit{limiting}) signal cross-section compatible with the observed data spectrum, such that any theory resulting in a signal cross-section above the limit is excluded with 95\% confidence. 
The limit is determined from a maximum likelihood test statistic \cite{likelihood}, which determines the likelihood of observing the given data spectrum using the background hypothesis, signal hypothesis, and uncertainty parameters.
Compatibility of the signal model with the observed distribution is tested by generating pseudo-data based on the background estimation and including varying amounts of signal.
Through analysis of these pseudo-data experiments, the maximum number of signals events that is compatible with the observed data distribution can be determined.
The 95\% confidence level is enforced by dictating that the number of signal events must be compatible with the observed data within 2$\sigma$ of uncertainty.

The final limits on the $Z'$ cross section after the implementation of the systematic uncertainties are shown in Figure~\ref{fig:unblinded_limits_syst}. 
\begin{figure}[!htbp]
\centering
   \includegraphics[width=0.95\textwidth]{figures/results/final_limits}
    \caption{Expected and observed 95\% CL limits in the unblinded SR, as a function of $Z'$ masses for \rinv=0.2 (top left), 0.4 (top right), 0.6 (bottom left), 0.8 (bottom right); no systematics.
    \label{fig:unblinded_limits_syst}}
\end{figure}
Exclusion of the theoretical model is observed for the 2000 GeV $Z'$ mass point for all \rinv~ values.
We are unable to exclude the highest mass points due to their low theoretical cross section, and relatively high spurious signal uncertainty.
The most mass points are excluded for \rinv~= 0.2, which excludes $Z'$ masses up to 3500 GeV.

%--------------------------------------------------------
\section{Discovery Result}
\label{sec:results_svj}
Figure~\ref{fig:unblinded_antelope_masked} shows the unblinded \mt~spectrum in the Discovery SR with a background-only fit, and the resulting BumpHunter test.
The polynomial fit is successful and has a B-only p-value of 0.74, indicating the data is compatible with the background hypothesis. 
The BumpHunter test gives a p-value of 0.8098, indicating no significant excess.
The maxiumum local significance is 0.877$\sigma$. 
\begin{figure}[!htbp]
\centering
   \includegraphics[width=0.95\textwidth]{figures/results/unblinded_antelope_unmasked}
    \caption{\mt~in the unblinded ANTELOPE SR with a background-only fit (p-value = 0.74), left. BumpHunter test selecting the most significant data excess with a p-value of 0.8098, right.
    \label{fig:unblinded_antelope_masked}}
\end{figure}

Because there is no specific signal interpretation for the Discovery region and both the polynomial fit and BH analysis are entirely data driven, there are no systematics to consider in the interpretation of the BH result.
%To further characterize the ANTELOPE \mt~spectrum, the bin interval of most significant deviation as identified by the BumpHunter test is masked, and the fit is rerun.
%The resulting polynomial fit has an improved p-value of 0.86, as anticipated, indicating good fit response (Figure~\ref{fig:unblinded_antelope_masked}.
%\begin{figure}[!htbp]
%\centering
%   \includegraphics[width=0.5\textwidth]{figures/results/unblinded_antelope_masked}
%    \caption{\mt~in the unblinded and BH masked ANTELOPE SR with a new background-only fit (p-value = 0.86, indicating improved compatibility), left. BumpHunter test selecting the most significant data excess with a p-value of 0.8822 (eg. less significant than unmasked).
%    \label{fig:unblinded_antelope_unmasked}}
%\end{figure}

