%Abstract Page

\begin{titlepage}
\begin{center}

\vspace*{1.5\baselineskip}
\textbf{\large Abstract}
 \vspace*{1\baselineskip}

Semi-Supervised Learning for Semi-Visible Jets: A Search for Dark Matter Jets at the LHC with the ATLAS Detector

 \vspace*{1\baselineskip}

Elena Laura Busch

 \vspace*{1\baselineskip}

\end{center}
\begin{flushleft}
\hspace{10mm}A search is presented for hadronic signatures of a strongly-coupled hidden dark sector, accessed via resonant production of a $Z'$ mediator. The analysis uses 139 $\text{fb}^{-1}$~of proton-proton collision data collected by the ATLAS experiment during Run 2 of the LHC. The $Z'$ mediator decays to two dark quarks, which each hadronize and decay to showers containing both dark and Standard Model particles; these showers are termed ``semi-visible'' jets. The final state expects missing energy aligned with one of the jets, a topology that is ignored by most dark matter searches. A supervised machine learning method is used to select these dark showers and reject the dominant background of mis-measured multijet events. A complementary semi-supervised anomaly detection approach preserves broad sensitivity to a variety of strongly coupled dark matter models. A resonance search is performed by fitting the transverse mass spectrum with a polynomial background estimation function. Results are presented as limits on the production cross section of the $Z'$, parameterized by the fraction of invisible particles in the decay and the $Z'$ mass. No structure in the transverse mass spectrum compatible with the signal hypothesis is observed. $Z'$ mediator masses from ranging from 2.0 TeV to 3.5 TeV are excluded at the 95\% confidence level.

\end{flushleft}
\vspace*{\fill}
\end{titlepage}
